\documentclass[a4paper, 12pt]{article}
\usepackage{amsmath}
\usepackage{dsfont}
\usepackage[utf8]{inputenc}
\usepackage{graphicx}
\usepackage[left=2cm, right=2cm, bottom=3cm, top=2cm]{geometry}
\usepackage{natbib}
\usepackage{microtype}

\newcommand{\hypers}{\boldsymbol{\alpha}}
\newcommand{\params}{\boldsymbol{\theta}}
\newcommand{\data}{\boldsymbol{x}}
\newcommand{\info}{\boldsymbol{M}}

\title{Crystals}
\author{Brendon J. Brewer}
\date{}

\begin{document}
\maketitle

%\abstract{\noindent Abstract}

% Need this after the abstract
\setlength{\parindent}{0pt}
\setlength{\parskip}{8pt}

\section{The prior information}

The joint prior distribution for the hyperparameters, parameters, and data,
is written $p(\hypers, \params, \data | \info)$, and is usually factorised
in this way:
\begin{align}
p(\hypers, \params, \data | \info) &=
    p(\hypers | \info)p(\params | \hypers, \info)
    p(\data | \params, \hypers, \info)\\
    &= p(\hypers | \info)p(\params | \hypers, \info)
    p(\data | \params, \info)
\end{align}
where the first step is true in general by the product rule, and the second
step assumes that knowing the parameters would make the hyperparameters
irrelevant when predicting what data would be observed.

All of the assumptions are given in Table~\ref{tab:priors}.

\begin{table}
\centering
\begin{tabular}{|lll|}
\hline
{\bf Quantity}      &   {\bf Meaning}   &  {\bf Prior distribution}\\
\hline
{\em Hyperparameters} & &\\
\hline
{\em Parameters}& &\\
\hline
{\em Data}&&\\
\hline
$\{y_1, y_2, ..., y_n\}$  &   Measurements    & $t$(, ,)\\
\hline
{\em Prior information}&&\\
\hline
$n$ & Number of measurements & given\\
$\{x_1, x_2, ..., x_n\}$  & $x$-values of measurements & given \\

\hline
\end{tabular}
\caption{\label{tab:priors}}
\end{table}



\bibliographystyle{plainnat}
\bibliography{references}

\end{document}

